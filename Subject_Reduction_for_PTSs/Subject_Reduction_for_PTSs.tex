\documentclass[reqno, twoside]{article}
\input{preamble.sty}
\input{macros.sty}

\usepackage{changepage}

\begin{document}
    \title{\textbf{\normalsize\MakeUppercase{Subject Reduction for Pure Type Systems}}}
    \author{\small Charlotte Marchal | 261031516\ \ \ \ Dashiell Rich | 261002837\\\small Milton Rosenbaum | 260972050\ \ \ \ Zhaoshen Zhai | 261003108}
    \date{}
    \maketitle
    \freefootnote{\textit{Date}: \today.}
    \freefootnote{Extended abstracted for the final project for \textsc{Comp527: Logic and Computation} taught by Professor Brigitte Pientka.}
    \freefootnote{Link to presentation: \TODO}

    \begin{center}
        \vspace{-0.3in}
        \begin{minipage}{0.85\textwidth}\setstretch{0.8}
            {\footnotesize{\textsc{Abstract.}} Following \cite{GN91} and \cite{Bar91}, we study the basics of \textit{pure type systems}, which abstract many of the constructs found in the eight systems of the \textit{$\lambda$-cube}. We start with a brief introduction to the systems of the $\lambda$-cube, discuss their expressive power, and introduce pure type systems as a unifying framework in which they can be studied. We then give a detailed proof of subject reduction for arbitrary pure type systems.}
        \end{minipage}
    \end{center}

    \subsection*{Introduction}

    Subject reduction is a crucial property of a type system that guarantees its `computational consistency' by ensuring that reductions of a well-typed expression remains well-typed, and which supports the slogan that `well-typed programs do not go wrong'. It is thus desirable that we can prove it uniformly across many different type systems, and this is the goal of the present note.

    To this end, \TODO[review $\lambda$-cube and the dependencies.]

    \begin{definition}
        A \textit{pure type system} is a tuple $\sigma\coloneqq(\mc{C},\mc{V},\mc{S},\mc{A},\mc{R})$ consisting of a set $\mc{C}$ of \textit{constants}, a set $\mc{V}$ of \textit{variables}, a set $\mc{S}\subeq\mc{C}$ of \textit{sorts}, a set $\mc{A}\subeq\mc{C}^2$ of \textit{axioms}, and a set $\mc{R}\subeq\mc{S}^3$ of \textit{rules}.
    \end{definition}

    \begin{notation}
        Throughout, let $\sigma$ be denote an arbitrary pure type system.
    \end{notation}

    \TODO[link this with the $\lambda$-cube by interpreting $\lambda_\to$ as a PTS.]

    \begin{definition}
        The collection of \textit{$\sigma$-pseudoterms} is defined by $T\coloneqq\mc{V}\,|\,\mc{C}\,|\,(T\,T)\,|\,(\lambda\mc{V}\!:\!T.T)\,|\,(\Pi\mc{V}\!:\!T.T)$. Pairs $(A,B)\in T^2$ are called \textit{$\sigma$-assignments}, written $A\!:\!B$, and a finite sequence thereof is called a \textit{$\sigma$-pseudocontext}.
    \end{definition}

    \begin{definition}
        The \textit{$\beta$-reduction} relation is the least relation on $\sigma$-terms satisfying the following for all $\sigma$-terms $A,A',A''$: the \textit{principal reduction rule} $(\lambda x\!:\!A.A')A''\rightarrow_\beta A'[A''/x]$, and the \textit{congruence rules} $A\,A'\succ A\,A''$, $A'\,A\succ A''\,A$, $\lambda x\!:\!A.A'\succ\lambda x\!:\!A.A''$, $\lambda x\!:\!A'.A\succ\lambda x\!:\!A''.A$, $\Pi x\!:\!A.A'\succ\Pi x\!:\!A.A''$, and $\Pi x\!:\!A'.A\succ\Pi x\!:\!A''.A$.
    \end{definition}

    \begin{notation}
        We write $\twoheadrightarrow_\beta$ for the reflexive and transitive closure of $\rightarrow_\beta$, and $=_\beta$ for the equivalence relation generated by $\twoheadrightarrow_\beta$. A $\sigma$-term of the form $(\lambda x\!:\!A.A')A''$ is called a \textit{$\beta$-redex}.
    \end{notation}

    \begin{definition}
        Let $\Gamma$ be a $\sigma$-pseudocontext and let $M,N$ be $\sigma$-pseudoterms. We say that \textit{$\Gamma$ proves $M\!:\!N$}, and write $\Gamma\proves M\!:\!N$, if there is a finite well-founded tree $\mc{D}$, called a \textit{derivation}, such that the following hold.
        \begin{enumerate}
            \item Vertices of $\mc{D}$ are of the form $\Delta\proves A\!:\!B$, where $A$ and $B$ are $\sigma$-pseudoterms and $\Delta$ is a $\sigma$-pseudocontext.
                \vspace{-0.20in}
            \item The root of $\mc{D}$ is $\Gamma\proves M\!:\!N$ and the leaves of $\mc{D}$ are instances of $\proves c\!:\!c'$, where $(c,c')\in\mc{A}$.
                \vspace{-0.05in}
            \item Each interior vertex of $\mc{D}$ is a conclusion of an \textit{inference rule}, whose successors are exactly the premises.
        \end{enumerate}
        The inference rules of $\sigma$ are as follows. Below, $s\in\mc{S}$, $x\in\mc{V}\comp\dom\Gamma$, $(s_1,s_2,s_3)\in\mc{R}$, and $C=_\beta C'$.
        \hfuzz16px\begin{adjustwidth}{-8pt}{0pt}\vspace{-0.1in}
        {\small\begin{equation*}
            \begin{gathered}
                \infer[\mathsc{Init}]{\Gamma,x\!:\!A\proves x\!:\!A}{\Gamma\proves A\!:\!s}\ \ \ \ 
                \infer[\mathsc{Weak}]{\Gamma,x\!:\!A\proves B\!:\!C}{\Gamma\proves A\!:\!s & \Gamma\proves B\!:\!C}\ \ \ \ 
                \infer[\mathsc{Conv}]{\Gamma\proves B\!:\!C'}{\Gamma\proves B\!:\!C & \Gamma\proves C'\!:\!s}\ \ \ \ 
                \infer[\Pi\mathsc{-rule}]{\Gamma\proves(\Pi x\!:\!B_1.B_2)\!:\!s_3}{\Gamma\proves B_1\!:\!s_1 & \Gamma,x\!:\!B_1\proves B_2\!:\!s_2}\\
                \infer[\lambda\mathsc{-rule}]{\Gamma\proves(\lambda x\!:\!B_1.C)\!:\!(\Pi x\!:\!B_1.B_2)}{\Gamma\proves B_1\!:\!s_1 & \Gamma,x\!:\!B_1\proves B_2\!:\!s_2 & \Gamma,x\!:\!B_1\proves C\!:\!B_2}\ \ \ \ 
                \infer[\mathsc{App}]{\Gamma\proves B_1\,B_2\!:\!C_2[B_2/x]}{\Gamma\proves B_1\!:\!(\Pi x\!:\!C_1.C_2) & \Gamma\proves B_2\!:\!C_1}
            \end{gathered}
        \end{equation*}}
        \end{adjustwidth}
    \end{definition}

    \begin{definition}
        If $\Gamma\proves A\!:\!B$, then $\Gamma$ is a \textit{$\sigma$-context} and $A,B$ are \textit{$\sigma$-terms}.
    \end{definition}

    \begin{lemma}[Substitution Lemma; \cite{GN91}*{Lemma 17}]\label{lem:substitution}
        Let $\Gamma$ and $\Gamma_1,y\!:\!A,\Gamma_2$ be $\sigma$-contexts and let $A,M,N,P$ be $\sigma$-terms. If $\Gamma_1,y\!:\!A,\Gamma_2\proves M\!:\!N$ and $\Gamma\proves P\!:\!A$, then $(\Gamma_1,\Gamma_2)[P/y]\proves M[P/y]\!:\!N[P/y]$.
    \end{lemma}

    \begin{lemma}[Stripping Lemma; \cite{GN91}*{Lemma 19}]\label{lem:stripping}
        Let $\Gamma$ be a $\sigma$-context and let $M,N,P$ be $\sigma$-terms.
        \begin{enumerate}
            \item If $\Gamma\proves c\!:\!P$ where $c\in\mc{C}$, then $P=_\beta c'$ and $(c,c')\in\mc{A}$ for some $c'\in\mc{C}$.
                \vspace{-0.05in}
            \item If $\Gamma\proves x\!:\!P$ where $x\in\mc{V}$, then $P=_\beta Q$ for some $\sigma$-term $Q$ such that $(x\!:\!Q)\in\Gamma$.
                \vspace{-0.05in}
            \item If $\Gamma\proves(\Pi x\!:\!M.N)\!:\!P$, then $\Gamma\proves M\!:\!s_1$, $\Gamma,x\!:\!M\proves N\!:\!s_2$, and $P=_\beta s_3$ for some $(s_1,s_2,s_3)\in\mc{R}$.
                \vspace{-0.05in}
            \item If $\Gamma\proves(\lambda x\!:\!M.N)\!:\!P$, then $\Gamma\proves M\!:\!s_1$, $\Gamma,x\!:\!M\proves Q\!:\!s_2$, $\Gamma,x\!:\!M\proves N\!:\!Q$, $\Gamma\proves P\!:\!s_3$, and $P=_\beta\Pi x\!:\! M.Q$ for some $(s_1,s_2,s_3)\in\mc{R}$ and $\sigma$-term $Q$.
                \vspace{-0.05in}
            \item If $\Gamma\proves M\,N\!:\!P$, then $\Gamma\proves M\!:\!(\Pi x\!:\!A.B)$, $\Gamma\proves N\!:\!A$, and $P=_\beta B[N/x]$ for some $\sigma$-terms $A$ and $B$.
        \end{enumerate}
    \end{lemma}

    \begin{theorem}[Subject Reduction; \cite{GN91}*{Lemma 22}]
        Let $\Gamma,\Gamma'$ be $\sigma$-contexts and let $M,M',N$ be $\sigma$-terms.
        \begin{enumerate}
            \item If $\Gamma\proves M\!:\!N$ and $M\twoheadrightarrow_\beta M'$, then $\Gamma\proves M'\!:\! N$.
                \vspace{-0.05in}
            \item If $\Gamma\proves M\!:\!N$ and $\Gamma\twoheadrightarrow_\beta\Gamma'$, then $\Gamma'\proves M\!:\! N$.
        \end{enumerate}
    \end{theorem}
    \begin{proof}
        We proceed by simultaneous induction on the derivation $\mc{D}:\Gamma\proves M\!:\!N$ when $M\rightarrow_\beta M'$ and $\Gamma\rightarrow_\beta\Gamma'$; the general case follows by iteration. We first prove (1), and split into cases with similar proofs.
        \begin{itemize}\small\vspace{-0.05in}
            \item If $\mc{D}$ ends with \textsc{Init}, then there is no redex in $M$. If $\mc{D}$ ends with \textsc{Conv}, then there are derivations $\mc{D}_1:\Gamma\proves M\!:\!N'$ and $\mc{D}_2:\Gamma\proves N'\!:\! s$ for some $s\in\mc{S}$ and some $\sigma$-term $N'$ such that $N'=_\beta N$. By $\mathrm{IH}_1$, we have $\Gamma\proves M'\!:\!N'$, on which \textsc{Conv} with $\mc{D}_2$ gives $\Gamma\proves M'\!:\!N$. The case when $\mc{D}$ ends with \textsc{Weak} is similar.
                \vspace{-0.05in}
            \item If $\mc{D}$ ends with $\Pi\mathsc{-rule}$, say with $M=\Pi x\!:\!B_1.B_2$, then the Stripping Lemma furnish some $(s_1,s_2,s_3)\in\mc{R}$ and derivations $\mc{D}_1:\Gamma\proves B_1\!:\!s_1$ and $\mc{D}_2:\Gamma,x\!:\!B_1\proves B_1\!:\!s_2$ such that $N=_\beta s_3$. By definition of $\rightarrow_\beta$, two cases occur: if there is a $\sigma$-term $B_1'$ such that $B_1\rightarrow_\beta B_1'$, then by $\mathrm{IH}_1$ on $\mc{D}_1$, we have $\mc{D}'_1:\Gamma\proves B_1'\!:\!s_1$. Moreover, $\mathrm{IH}_2$ on $\mc{D}_2$ gives $\mc{D}'_2:\Gamma,x\!:\!B_1'\proves B_2\!:\!s_2$, so applying $\Pi\mathsc{-rule}$ on $\mc{D}'_1$ and $\mc{D}'_2$ gives $\Gamma\proves(\Pi x\!:\!B_1'.B_2)\!:\!s_3$, on which \textsc{Conv} gives $\Gamma\proves(\Pi x\!:\!B_1'.B_2)\!:\!N$. The second case when $B_2\rightarrow_\beta B_2'$ for some $\sigma$-term $B_2'$ is the same (in fact, easier). The case when $\mc{D}$ ends with $\lambda\mathsc{-rule}$ is similar (and again has two subcases).
                \vspace{-0.05in}
            \item If $\mc{D}$ ends with \textsc{App}, say with $M=B_1\,B_2$, then reductions within either $B_1$ or $B_2$ are trivial. Thus, we can take $x\in\mc{V}\comp\dom\Gamma$ such that $B_1=\lambda x\!:\!A_1.A_2$, and assume $M=(\lambda x\!:\!A_1.A_2)B_2\rightarrow_\beta A_2[B_2/x]$. The Stripping Lemma then furnish $\sigma$-terms $C_1$ and $C_2$ such that $N=_\beta C_2[B_2/x]$ and derivations $\mc{D}_1:\Gamma\proves(\lambda x\!:\!A_1.A_2)\!:\!(\Pi x\!:\!C_1.C_2)$ and $\mc{D}_2:\Gamma\proves B_2\!:\!C_1$. Again, the Stripping Lemma applied to $\mc{D}_1$ then furnish $(s_1,s_2,s_3)\in\mc{R}$, a $\sigma$-term $C_2'$ such that $\Pi x\!:\!C_1.C_2=_\beta\Pi x\!:\!A_1.C_2'$, and derivations $\mc{E}_1:\Gamma\proves A_1\!:\!s_1$, $\mc{E}_2:\Gamma,x\!:\!A_1\proves C'_2\!:\!s_2$, and $\mc{E}_3:\Gamma,x\!:\!A_1\proves A_2\!:\!C_2'$. Observe that $A_1=_\beta C_1$, so \textsc{Conv} on $\mc{D}_2$ and $\mc{E}_1$ gives $\mc{D}_0:\Gamma\proves B_2\!:\!A_1$, and using the Substitution Lemma with $(\mc{D}_0,\mc{E}_2)$ and $(\mc{D}_0,\mc{E}_3)$ give $\mc{E}_2':\Gamma\proves C_2'[B_2/x]:s_2$ and $\mc{E}_3':\Gamma\proves A_2[B_2/x]\!:\!C_2'[B_2/x]$; note that $\Gamma[B_2/x]=\Gamma$ since $x\not\in\dom\Gamma$. Finally, since $C_2=_\beta C_2'$ and $N=_\beta C_2[B_2/x]$, applying \textsc{Conv} on $\mc{E}_2'$ and $\mc{E}_3'$ gives $\Gamma\proves A_2[B_2/x]\!:\!N$.
        \end{itemize}
    \end{proof}

    \appendix

    \begin{bibdiv}
        \begin{biblist}*{labels={alphabetic}}
            \bibselect{bibliography}
        \end{biblist}
    \end{bibdiv}
\end{document}
