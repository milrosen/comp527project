\documentclass[reqno, twoside]{article}
\usepackage[T1]{fontenc}
\usepackage{amsfonts, amsmath, amssymb, amsthm, amsrefs}
\usepackage{mathtools, mathrsfs, dsfont, nicematrix}
\usepackage{tikz, tikz-3dplot, tikzpagenodes, graphicx, xcolor}
\usepackage{geometry, mdframed, titlesec, fancyhdr, caption, subcaption}
\usepackage{enumitem, bookmark, xifthen, setspace}

\usetikzlibrary{matrix, positioning, patterns, decorations.markings, arrows, arrows.meta, backgrounds, math, cd}
\tikzset{->-/.style={decoration={ markings, mark=at position #1 with {\arrow{>}}},postaction={decorate}}}

\definecolor{darkBlue}{RGB}{0, 0, 138}
\definecolor{darkGreen}{RGB}{0, 160, 0}
\definecolor{lightGray}{RGB}{128, 128, 128}

\hypersetup{colorlinks=true, allcolors=darkBlue}
\newgeometry{margin = 1in}
\pagestyle{fancy}\fancyhead{}\fancyfoot{}\headsep=15pt\renewcommand{\headrulewidth}{0pt}
\fancyhead[RO, LE]{\scriptsize\thepage}
\fancyhead[CE]{\scriptsize\MakeUppercase{COMP532: Logic and Computation}}
\fancyhead[CO]{\scriptsize\MakeUppercase{Subject Reduction for Pure Type Systems}}

\titleformat{name=\section}{}{\thetitle.}{0.8em}{\centering\scshape}
\titleformat{name=\subsection}[runin]{}{\thetitle.}{0.5em}{\bfseries}[.]
\titleformat{name=\subsubsection}[runin]{}{\thetitle.}{0.5em}{\itshape}[.]

\newtheorem{mainTheorem}{Theorem}\renewcommand{\themainTheorem}{\Alph{mainTheorem}}
\newtheorem{theorem}{Theorem}\newtheorem*{theorem*}{Theorem}
\newtheorem{proposition}[theorem]{Proposition}\newtheorem*{proposition*}{Proposition}
\newtheorem{lemma}[theorem]{Lemma}\newtheorem*{lemma*}{Lemma}
\newtheorem{claim}[theorem]{Claim}\newtheorem*{claim*}{Claim}
\newtheorem{thesis}[theorem]{Thesis}\newtheorem*{thesis*}{Thesis}
\newtheorem{corollary}[theorem]{Corollary}\newtheorem*{corollary*}{Corollary}

\theoremstyle{definition}{
    \newtheorem{example}[theorem]{Example}\newtheorem*{example*}{Example}
    \newtheorem{definition}[theorem]{Definition}\newtheorem*{definition*}{Definition}
    \newtheorem{remark}[theorem]{Remark}\newtheorem*{remark*}{Remark}
    \newtheorem{notation}[theorem]{Notation}\newtheorem*{notation*}{Notation}
    \newtheorem{observation}[theorem]{Observation}\newtheorem*{observation*}{Observation}
    \newtheorem{axiom}[theorem]{Axiom}\newtheorem*{axiom*}{Axiom}
    \newtheorem{question}[theorem]{Question}\newtheorem*{question*}{Question}
}

\newmdenv[topline=false, bottomline=false, rightline=false, skipabove=4pt, skipbelow=4pt, linewidth=0.75pt]{leftbar}
\newcommand\freefootnote[1]{\begin{NoHyper}\renewcommand\thefootnote{}\footnote{#1}\addtocounter{footnote}{-1}\end{NoHyper}}

% Operators
\newcommand{\id}{\operatorname{id}}
\newcommand{\im}{\operatorname{im}}
\newcommand{\rk}{\operatorname{rk}}
\newcommand{\ch}{\operatorname{ch}}
\newcommand{\tr}{\operatorname{tr}}
\newcommand{\tp}{\operatorname{tp}}
\newcommand{\ot}{\operatorname{ot}}
\newcommand{\ad}{\operatorname{ad}}
\newcommand{\cf}{\operatorname{cf}}
\newcommand{\Id}{\operatorname{Id}}
\newcommand{\Th}{\operatorname{Th}}
\newcommand{\Cn}{\operatorname{Cn}}
\newcommand{\Bl}{\operatorname{Bl}}
\newcommand{\Cl}{\operatorname{Cl}}
\newcommand{\Ad}{\operatorname{Ad}}
\newcommand{\LT}{\operatorname{LT}}
\newcommand{\dom}{\operatorname{dom}}
\newcommand{\ran}{\operatorname{ran}}
\newcommand{\cdm}{\operatorname{cdm}}
\newcommand{\sgn}{\operatorname{sgn}}
\newcommand{\lcm}{\operatorname{lcm}}
\newcommand{\ord}{\operatorname{ord}}
\newcommand{\cvx}{\operatorname{cvx}}
\newcommand{\Aut}{\operatorname{Aut}}
\newcommand{\Inn}{\operatorname{Inn}}
\newcommand{\Out}{\operatorname{Out}}
\newcommand{\End}{\operatorname{End}}
\newcommand{\Mat}{\operatorname{Mat}}
\newcommand{\Obj}{\operatorname{Obj}}
\newcommand{\Hom}{\operatorname{Hom}}
\newcommand{\Tor}{\operatorname{Tor}}
\newcommand{\Ext}{\operatorname{Ext}}
\newcommand{\Ann}{\operatorname{Ann}}
\newcommand{\Sym}{\operatorname{Sym}}
\newcommand{\Alt}{\operatorname{Alt}}
\newcommand{\Cov}{\operatorname{Cov}}
\newcommand{\Orb}{\operatorname{Orb}}
\newcommand{\Sat}{\operatorname{Sat}}
\newcommand{\Thm}{\operatorname{Thm}}
\newcommand{\Der}{\operatorname{Der}}
\newcommand{\Def}{\operatorname{Def}}
\newcommand{\Age}{\operatorname{Age}}
\newcommand{\Div}{\operatorname{Div}}
\newcommand{\Lie}{\operatorname{Lie}}
\newcommand{\Rep}{\operatorname{Rep}}
\newcommand{\Bil}{\operatorname{Bil}}
\newcommand{\Ind}{\operatorname{Ind}}
\newcommand{\Res}{\operatorname{Res}}
\newcommand{\Cay}{\operatorname{Cay}}
\newcommand{\Min}{\operatorname{Min}}
\newcommand{\Con}{\operatorname{Con}}
\newcommand{\PGL}{\operatorname{PGL}}
\newcommand{\rank}{\operatorname{rank}}
\newcommand{\proj}{\operatorname{proj}}
\newcommand{\diag}{\operatorname{diag}}
\newcommand{\eval}{\operatorname{eval}}
\newcommand{\cont}{\operatorname{cont}}
\newcommand{\diam}{\operatorname{diam}}
\newcommand{\mult}{\operatorname{mult}}
\newcommand{\trcl}{\operatorname{trcl}}
\newcommand{\supp}{\operatorname{supp}}
\newcommand{\conv}{\operatorname{conv}}
\newcommand{\Core}{\operatorname{Core}}
\newcommand{\Term}{\operatorname{Term}}
\newcommand{\Taut}{\operatorname{Taut}}
\newcommand{\Sent}{\operatorname{Sent}}
\newcommand{\Skew}{\operatorname{Skew}}
\newcommand{\Frac}{\operatorname{Frac}}
\newcommand{\Stab}{\operatorname{Stab}}
\newcommand{\Meas}{\operatorname{Meas}}
\newcommand{\Diag}{\operatorname{Diag}}
\newcommand{\Diff}{\operatorname{Diff}}
\newcommand{\Isom}{\operatorname{Isom}}
\newcommand{\Area}{\operatorname{Area}}
\newcommand{\coker}{\operatorname{coker}}
\newcommand{\preim}{\operatorname{preim}}
\newcommand{\Graph}{\operatorname{Graph}}
\newcommand{\Axioms}{\operatorname{Axioms}}
\renewcommand{\Re}{\operatorname{Re}}
\renewcommand{\Im}{\operatorname{Im}}

% Math notations
% Set Theory, Category Theory, and Logic
    \newcommand{\fa}{\forall}
    \newcommand{\ex}{\exists}
    \newcommand{\iso}{\cong}
    \newcommand{\cat}[1]{\mathbf{#1}}
    \newcommand{\pow}{\mathcal{P}}
    \newcommand{\comp}{\setminus}
    \newcommand{\limp}{\multimap}
    \newcommand{\code}[2][]{\ulcorner#1#2#1\urcorner}
    \newcommand{\cprod}{\amalg}
    \newcommand{\eqnum}{\approx}
    \newcommand{\natiso}{\simeq}
    \newcommand{\proves}{\vdash}
    \newcommand{\forces}{\Vdash}
    \newcommand{\nproves}{\nvdash}
    \newcommand{\symdiff}{\bigtriangleup}
    \newcommand{\ladjoint}{\dashv}
    \newcommand{\radjoint}{\vdash}
    \renewcommand{\em}{\varnothing}
    \renewcommand{\vec}[1]{\bar{#1}}

% Analysis
    \newcommand{\BV}{BV}
    \newcommand{\del}{\partial}
    \newcommand{\abscont}{\ll}
    \newcommand{\esssup}{\operatorname{ess-sup}}
    \renewcommand{\d}{\mathrm{d}}

% Topology
    \newcommand{\htopeq}{\simeq}

% Group Theory
    \newcommand{\act}{\curvearrowright}
    \newcommand{\acted}{\curvearrowleft}
    \newcommand{\semiact}{\ltimes}
    \newcommand{\semiacted}{\rtimes}

% Number Theory
    \newcommand{\ndiv}{\nmid}
    \renewcommand{\div}{\,|\,}

% Misc
    \newcommand{\st}{:}
    \newcommand{\tpl}[1]{\l(#1\r)}
    \newcommand{\gen}[2][]{\ifthenelse{\isempty{#1}}{}{\l}\langle#2\ifthenelse{\isempty{#1}}{}{\r}\rangle}
    \newcommand{\slot}{-}
    \newcommand{\blob}{\bullet}
    \renewcommand{\l}{\left}
    \renewcommand{\r}{\right}
    \renewcommand{\bar}{\overline}

% Arrows
    \newcommand{\too}[2][]{\xlongrightarrow[#1]{#2}}
    \newcommand{\onto}{\twoheadrightarrow}
    \newcommand{\into}{\hookrightarrow}
    \newcommand{\intoo}[2][]{\lhook\joinrel\xlongrightarrow[#1]{#2}}
    \newcommand{\ontoo}[2][]{\xlongrightarrow[#1]{#2}\mathrel{\mkern-14mu}\rightarrow}
    \newcommand{\parto}{\rightharpoonup}
    \newcommand{\ratto}{\dashrightarrow}
    \newcommand{\incto}{\nearrow}
    \newcommand{\decto}{\searrow}
    \newcommand{\pathto}{\rightsquigarrow}

% Subobjects
    \newcommand{\sub}{\subset}
    \newcommand{\subs}{<}
    \newcommand{\sups}{>}
    \newcommand{\esub}{\prec}
    \newcommand{\esup}{\succ}
    \newcommand{\nsub}{\triangleleft}
    \newcommand{\nsup}{\triangleright}
    \newcommand{\subeq}{\subseteq}
    \newcommand{\subseq}{\leq}
    \newcommand{\supseq}{\geq}
    \newcommand{\esubeq}{\preceq}
    \newcommand{\esupeq}{\succeq}
    \newcommand{\nsubeq}{\trianglelefteq}
    \newcommand{\nsupeq}{\trianglerighteq}

% Number Systems
    \newcommand{\N}{\mathbb{N}}
    \newcommand{\Z}{\mathbb{Z}}
    \newcommand{\Q}{\mathbb{Q}}
    \newcommand{\R}{\mathbb{R}}
    \newcommand{\C}{\mathbb{C}}
    \newcommand{\F}{\mathbb{F}}
    \newcommand{\E}{\mathbb{E}}
    \newcommand{\A}{\mathbb{A}}
    \renewcommand{\H}{\mathbb{H}}
    \renewcommand{\S}{\mathbb{S}}
    \renewcommand{\P}{\mathbb{P}}

% LaTeX
% Fonts
    \newcommand{\mc}[1]{\mathcal{#1}}
    \newcommand{\ms}[1]{\mathscr{#1}}
    \newcommand{\mb}[1]{\mathbb{#1}}
    \newcommand{\mf}[1]{\mathfrak{#1}}
    \newcommand{\ds}[1]{\mathds{#1}}
    \newcommand{\bb}[1]{\mathbb{#1}}
    \newcommand{\mathsc}[1]{{\normalfont\textsc{#1}}}
    \renewcommand{\it}[1]{\textit{#1}}
    \renewcommand{\bf}[1]{\textbf{#1}}
    \renewcommand{\sf}[1]{\textsf{#1}}
    \renewcommand{\phi}{\varphi}
    \renewcommand{\epsilon}{\varepsilon}

% Meta
    \newcommand{\TODO}[1][]{{\color{red}\textbf{TODO}\ifthenelse{\isempty{#1}}{}{\textbf{:} #1}}}
    \newcommand{\qedin}{\tag*{$\blacksquare$}}
    \newcommand{\qedlemin}{\tag*{$\square$}}
    \newcommand{\qedlem}{\phantom\qedhere\hfill$\square$}
    \renewcommand{\qed}{\phantom\qedhere\hfill$\blacksquare$}


\usepackage{changepage}

\begin{document}
    \title{\textbf{\normalsize\MakeUppercase{Subject Reduction for Pure Type Systems}}}
    \author{\small Charlotte Marchal | 261031516\ \ \ \ Dashiell Rich | 261002837\\\small Milton Rosenbaum | 260972050\ \ \ \ Zhaoshen Zhai | 261003108}
    \date{}
    \maketitle
    \freefootnote{\textit{Date}: \today.}
    \freefootnote{Extended abstracted for the final project for \textsc{Comp527: Logic and Computation} taught by Professor Brigitte Pientka.}
    \freefootnote{Link to presentation: \url{https://www.youtube.com/watch?v=6golOB-Mn3U}.}

    \begin{center}
        \vspace{-0.3in}
        \begin{minipage}{0.85\textwidth}\setstretch{0.8}
            {\footnotesize{\textsc{Abstract.}} Following \cite{GN91} and \cite{Bar91}, we study the basics of \textit{pure type systems}, which abstract many of the constructs found in the eight systems of the \textit{$\lambda$-cube}. We start with a brief introduction to the systems of the $\lambda$-cube, discuss their expressive power, and introduce pure type systems as a unifying framework in which they can be studied. We then give a detailed proof of subject reduction for arbitrary pure type systems.}
        \end{minipage}
    \end{center}

    \subsection*{Introduction}
    
    Subject reduction is a crucial property of a type system that guarantees its `computational consistency' by ensuring that reductions of a well-typed expression remains well-typed, and also supports the slogan that `well-typed programs do not go wrong'. It is thus desirable that it can be stated and proven uniformly across many different type systems, and this is the goal of the present note.

    To this end, we follow \cite{Bar91} and work within the framework of \textit{pure type systems}, which is an abstraction of type systems based on the idea of `dependencies' between terms and types. In the simply-typed $\lambda$-calculus $\lambda_\rightarrow$, terms depend on terms: $F\,x$ is a term depending on the term $x$, and we can abstract this dependency to a function $\lambda x\!:\!\alpha_1.(F\,x)$ of type $\alpha_1\rightarrow\alpha_2$, where $F\,x\!:\!\alpha_2$. The other three ways that terms and types can be mutually dependent are present in other type systems, which all extend $\lambda_\rightarrow$.
    \begin{itemize}
        \item In the polymorphic $\lambda$-calculus $\lambda2$, the term $I_\alpha\coloneqq\lambda x\!:\!\alpha.x$ depends on the type $\alpha$, and this abstracts to $\lambda\alpha\!:\!\ast.I_\alpha$ of type $\fa\alpha\!:\!\ast.(\alpha\rightarrow\alpha)$. Here, `$\alpha\!:\!\ast$' formalizes `$\alpha$ is a type' within $\lambda2$.
            \vspace{-0.05in}
        \item In the (weak) higher-order $\lambda$-calculus $\lambda\underline{\omega}$, the type $\alpha\rightarrow\alpha$ depends on the type $\alpha$, and this abstracts to a \textit{constructor} $\lambda\alpha\!:\!\ast.(\alpha\rightarrow\alpha)$ of \textit{kind} $\ast\to\ast$. Similarly, in the $\lambda$-calculus $\lambda\mathrm{P}$ of dependent types, the type $\alpha_1^n\to\alpha_2$ depends on the term $n$, and abstracts to a constructor $\lambda n\!:\!\N.(\alpha_1^n\to\alpha_2)$ of kind $\N\to\ast$.
    \end{itemize}
    Assuming that the type systems that we care about all extend $\lambda_\rightarrow$, and hence terms can depend on terms, we obtain a total of $8=2^3$ type systems with all possible combinations of dependencies, called the \textit{$\lambda$-cube}:
    \vspace{-0.05in}
    \begin{center}
        \begin{minipage}{0.3\textwidth}
            \vspace{-0.15in}
            \begin{equation*}
                \begin{tikzcd}[row sep=scriptsize, column sep=0.1in]
                    & \lambda\omega \arrow[dl, -] \arrow[rr, -] \arrow[dd, -] & & \lambda\mathrm{C} \arrow[dl, -] \arrow[dd, -] \\
                    \lambda2 \arrow[rr, crossing over, -] \arrow[dd, -] & & \lambda\mathrm{P}2 \\
                    & \lambda\underline{\omega} \arrow[dl, -] \arrow[rr, -] & & \lambda\mathrm{P}\underline{\omega} \arrow[dl, -] \\
                    \lambda_\rightarrow \arrow[rr, -] & & \lambda\mathrm{P} \arrow[from=uu, crossing over, -]
                \end{tikzcd}
                \vspace{-0.05in}
            \end{equation*}
        \end{minipage}
        \begin{minipage}{0.5\textwidth}\small
            \begin{center}
                \begin{tabular}{c|cccc}
                    Type system & \multicolumn{4}{c}{Dependencies} \\
                    \hline
                    $\lambda_\rightarrow$                                    & $(\ast,\ast)$ &                  &                     &                  \\
                    $\lambda2$                                               & $(\ast,\ast)$ & $(\square,\ast)$ &                     &                  \\
                    $\lambda\underline{\omega}$                              & $(\ast,\ast)$ &                  & $(\square,\square)$ &                  \\
                    $\lambda\omega=\lambda2\underline{\omega}$               & $(\ast,\ast)$ & $(\square,\ast)$ & $(\square,\square)$ &                  \\
                    $\lambda\mathrm{P}$                                      & $(\ast,\ast)$ &                  &                     & $(\ast,\square)$ \\
                    $\lambda\mathrm{P}2$                                     & $(\ast,\ast)$ & $(\square,\ast)$ &                     & $(\ast,\square)$ \\
                    $\lambda\mathrm{P}\underline{\omega}$                    & $(\ast,\ast)$ &                  & $(\square,\square)$ & $(\ast,\square)$ \\
                    $\lambda\mathrm{C}=\lambda\mathrm{P}2\underline{\omega}$ & $(\ast,\ast)$ & $(\square,\ast)$ & $(\square,\square)$ & $(\ast,\square)$ \\
                \end{tabular}
            \end{center}
        \end{minipage}
    \end{center}
    The systems $\lambda\mathrm{P}\underline{\omega}$, $\lambda\mathrm{P}2$, and $\lambda\omega\coloneqq\lambda2\underline{\omega}$ have three kinds of dependencies, while the strongest system of them all, the \textit{calculus of constructions} $\lambda\mathrm{C}\coloneqq\lambda\mathrm{P}2\underline{\omega}$, have all four kinds of dependencies.

    To explain the table on the right, we observe that due to the mutual dependencies of terms and types, it is no longer natural to separate their definitions. Instead, we propose a uniform object, called a \textit{pseudoterm}, and consider a judgement $M\!:\!N$ between pseudoterms $M$ and $N$. These pseudoterms include terms, types, and kinds, so this begs the question: what is $\ast$? If $\ast\!:\!\ast$, then one might encounter a Russell-like paradox of the `type of all types', so instead, we introduce a new symbol $\square$, the `sort of all kinds', and assert that $\ast\!:\!\square$. Similarly, $(\ast\to\ast)\!:\!\square$ and $(\N\to\ast)\!:\!\square$, so sorts also capture the identity of (higher order) type constructors.

    In the table, the notation $(s_1,s_2)$ means that inhabitants of $s_2$ can depend on those of $s_1$, and moreover, that we can abstract over inhabitants of $s_1$ and output those of $s_2$. For instance, the `$(\square,\ast)$' in $\lambda2$ indicate that terms (inhabitants of $\ast$) depend on types (inhabitants of $\square$), and that we can abstract over types and output terms (say, in $\lambda\alpha\!:\!\ast.I_\alpha$). Note that $(\ast\to\ast)\!:\!\square$, so this allows for higher-order polymorphism as well.

    \subsection*{Pure type systems}

    To delineate the hierarchy between terms, types, and kinds, one is naturally inclined to start abstractly and axiomatize the notion of a type system.

    \begin{definition}
        A \textit{pure type system} is a tuple $\sigma\coloneqq(\mc{C},\mc{V},\mc{S},\mc{A},\mc{R})$ consisting of a set $\mc{C}$ of \textit{constants}, a set $\mc{V}$ of \textit{variables}, a set $\mc{S}\subeq\mc{C}$ of \textit{sorts}, a set $\mc{A}\subeq\mc{C}^2$ of \textit{axioms}, and a set $\mc{R}\subeq\mc{S}^3$ of \textit{rules}.
    \end{definition}

    Intuitively, \textit{sorts} are universes imposing some sort (pun intended) of classification, \textit{constants} are symbols living in a sort/constant as dictated by \textit{axioms} (for instance, $0\!:\!\N$, $\N\!:\!\ast$, and $\ast\!:\!\square$), and the \textit{rules} restrict the formation of abstractions as motivated above. For the systems in the $\lambda$-cube, we write $(s_1,s_2)\coloneqq(s_1,s_2,s_2)$. See \cite{Bar91} or \cite{Bar92} for how the eight systems in the $\lambda$-cube can be interpreted as pure type systems.

    Let us now proceed by defining the necessary notions to state and prove subject reduction for an arbitrary pure type system $\sigma$, for which we will need two lemmas in \cite{GN91}. We also invite the reader to see how the following restrict to the standard notions/proofs for systems in the $\lambda$-cube.

    \begin{definition}
        The collection of \textit{$\sigma$-pseudoterms} is defined by $T\coloneqq\mc{V}\,|\,\mc{C}\,|\,(T\,T)\,|\,(\lambda\mc{V}\!:\!T.T)\,|\,(\Pi\mc{V}\!:\!T.T)$. Pairs $(A,B)\in T^2$ are called \textit{$\sigma$-assignments}, written $A\!:\!B$, and a finite sequence thereof is called a \textit{$\sigma$-pseudocontext}.
    \end{definition}

    \begin{definition}
        The \textit{$\beta$-reduction} relation is the least relation $\rightarrow_\beta$ on $\sigma$-terms satisfying the following rules for all $\sigma$-terms $A,A',A''$: $(\lambda x\!:\!A.A')A''\rightarrow_\beta A'[A''/x]$, and the \textit{congruence rules} $A\,A'\rightarrow_\beta A\,A''$, $A'\,A\rightarrow_\beta A''\,A$, $\lambda x\!:\!A.A'\rightarrow_\beta\lambda x\!:\!A.A''$, $\lambda x\!:\!A'.A\rightarrow_\beta\lambda x\!:\!A''.A$, $\Pi x\!:\!A.A'\rightarrow_\beta\Pi x\!:\!A.A''$, and $\Pi x\!:\!A'.A\rightarrow_\beta\Pi x\!:\!A''.A$.
    \end{definition}

    \begin{notation}
        We write $\twoheadrightarrow_\beta$ for the reflexive and transitive closure of $\rightarrow_\beta$, and $=_\beta$ for the equivalence relation generated by $\twoheadrightarrow_\beta$. A $\sigma$-term of the form $(\lambda x\!:\!A.A')A''$ is called a \textit{$\beta$-redex}. If $\Gamma\coloneqq(x_1\!:\!A_1,\dots,x_n\!:\!A_n)$ and $\Gamma'\coloneqq(x_1\!:\!A_1',\dots,x_n\!:\!A_n')$ are $\sigma$-pseudocontexts, we also write $\Gamma\rightarrow_\beta\Gamma$ if $A_i\rightarrow_\beta A_i'$ for each $i\leq n$.
    \end{notation}

    \begin{definition}\label{rules}
        Let $\Gamma$ be a $\sigma$-pseudocontext and let $M,N$ be $\sigma$-pseudoterms. We say that \textit{$\Gamma$ proves $M\!:\!N$}, and write $\Gamma\proves M\!:\!N$, if there is a finite well-founded tree $\mc{D}$, called a \textit{derivation}, such that the following hold.
        \begin{enumerate}
            \item Vertices of $\mc{D}$ are of the form $\Delta\proves A\!:\!B$, where $A$ and $B$ are $\sigma$-pseudoterms and $\Delta$ is a $\sigma$-pseudocontext.
                \vspace{-0.20in}
            \item The root of $\mc{D}$ is $\Gamma\proves M\!:\!N$ and the leaves of $\mc{D}$ are instances of $\proves c\!:\!c'$, where $(c,c')\in\mc{A}$.
                \vspace{-0.05in}
            \item Each interior vertex of $\mc{D}$ is a conclusion of an \textit{inference rule}, whose successors are exactly the premises.
        \end{enumerate}
        The inference rules of $\sigma$ are as follows. Below, $s\in\mc{S}$, $x\in\mc{V}\comp\dom\Gamma$, $(s_1,s_2,s_3)\in\mc{R}$, and $C=_\beta C'$.
        \hfuzz16px\begin{adjustwidth}{-8pt}{0pt}\vspace{-0.1in}
        {\small\begin{equation*}
            \begin{gathered}
                \infer[\mathsc{Init}]{\Gamma,x\!:\!A\proves x\!:\!A}{\Gamma\proves A\!:\!s}\ \ \ \ 
                \infer[\mathsc{Weak}]{\Gamma,x\!:\!A\proves B\!:\!C}{\Gamma\proves A\!:\!s & \Gamma\proves B\!:\!C}\ \ \ \ 
                \infer[\mathsc{Conv}]{\Gamma\proves B\!:\!C'}{\Gamma\proves B\!:\!C & \Gamma\proves C'\!:\!s}\ \ \ \ 
                \infer[\Pi\mathsc{-rule}]{\Gamma\proves(\Pi x\!:\!B_1.B_2)\!:\!s_3}{\Gamma\proves B_1\!:\!s_1 & \Gamma,x\!:\!B_1\proves B_2\!:\!s_2}\\
                \infer[\lambda\mathsc{-rule}]{\Gamma\proves(\lambda x\!:\!B_1.C)\!:\!(\Pi x\!:\!B_1.B_2)}{\Gamma\proves B_1\!:\!s_1 & \Gamma,x\!:\!B_1\proves B_2\!:\!s_2 & \Gamma,x\!:\!B_1\proves C\!:\!B_2}\ \ \ \ 
                \infer[\mathsc{App}]{\Gamma\proves B_1\,B_2\!:\!C_2[B_2/x]}{\Gamma\proves B_1\!:\!(\Pi x\!:\!C_1.C_2) & \Gamma\proves B_2\!:\!C_1}
            \end{gathered}
        \end{equation*}}
        \end{adjustwidth}
    \end{definition}

    \begin{definition}
        If $\Gamma\proves M\!:\!N$, then $\Gamma$ is called a \textit{$\sigma$-context} and $M,N$ are called \textit{$\sigma$-terms}.
    \end{definition}

    \begin{lemma}[Substitution Lemma; \cite{GN91}*{Lemma 17}]\label{lem:substitution}
        Let $\Gamma$ and $\Gamma_1,y\!:\!A,\Gamma_2$ be $\sigma$-contexts and let $A,M,N,P$ be $\sigma$-terms. If $\Gamma_1,y\!:\!A,\Gamma_2\proves M\!:\!N$ and $\Gamma\proves P\!:\!A$, then $(\Gamma_1,\Gamma_2)[P/y]\proves M[P/y]\!:\!N[P/y]$.
    \end{lemma}

    \begin{lemma}[Stripping Lemma; \cite{GN91}*{Lemma 19}]\label{lem:stripping}
        Let $\Gamma$ be a $\sigma$-context and let $M,N,P$ be $\sigma$-terms.
        \begin{enumerate}
            \item If $\Gamma\proves c\!:\!P$ where $c\in\mc{C}$, then $P=_\beta c'$ and $(c,c')\in\mc{A}$ for some $c'\in\mc{C}$.
                \vspace{-0.05in}
            \item If $\Gamma\proves x\!:\!P$ where $x\in\mc{V}$, then $P=_\beta Q$ for some $\sigma$-term $Q$ such that $(x\!:\!Q)\in\Gamma$.
                \vspace{-0.05in}
            \item If $\Gamma\proves(\Pi x\!:\!M.N)\!:\!P$, then $\Gamma\proves M\!:\!s_1$, $\Gamma,x\!:\!M\proves N\!:\!s_2$, and $P=_\beta s_3$ for some $(s_1,s_2,s_3)\in\mc{R}$.
                \vspace{-0.05in}
            \item If $\Gamma\proves(\lambda x\!:\!M.N)\!:\!P$, then $\Gamma\proves M\!:\!s_1$, $\Gamma,x\!:\!M\proves Q\!:\!s_2$, $\Gamma,x\!:\!M\proves N\!:\!Q$, $\Gamma\proves P\!:\!s_3$, and $P=_\beta\Pi x\!:\! M.Q$ for some $(s_1,s_2,s_3)\in\mc{R}$ and $\sigma$-term $Q$.
                \vspace{-0.05in}
            \item If $\Gamma\proves M\,N\!:\!P$, then $\Gamma\proves M\!:\!(\Pi x\!:\!C_1.C_2)$, $\Gamma\proves N\!:\!C_1$, and $P=_\beta C_2[N/x]$ for some $\sigma$-terms $C_1$ and $C_2$.
        \end{enumerate}
    \end{lemma}

    \begin{theorem}[Subject Reduction; \cite{GN91}*{Lemma 22}]\label{thm:subject_reduction}
        Let $\Gamma,\Gamma'$ be $\sigma$-contexts and let $M,M',N$ be $\sigma$-terms.
        \begin{enumerate}
            \item If $\Gamma\proves M\!:\!N$ and $M\twoheadrightarrow_\beta M'$, then $\Gamma\proves M'\!:\! N$.
                \vspace{-0.05in}
            \item If $\Gamma\proves M\!:\!N$ and $\Gamma\twoheadrightarrow_\beta\Gamma'$, then $\Gamma'\proves M\!:\! N$.
        \end{enumerate}
    \end{theorem}

    Let us close by mentioning that beyond providing uniform proofs for many statements in the $\lambda$-cube (see \cite{GN91} for a modular proof of strong normalization for systems in the $\lambda$-cube), pure type systems remain an active area of research nowadays. See \cite{Ter95}, \cite{BHS01}, and \cite{BG05} for surveys and extensions.

    \subsection*{Appendix}

    We sketch a proof of the Stripping Lemma, since it is crucial to the proof of subject reduction, and use it to provide the full proof of subject reduction for an arbitrary pure type system $\sigma$.

    \begin{proof}[Proof of Lemma \ref{lem:stripping}]
        Let $\mc{D}$ denote the derivation $\Gamma\proves A\!:\!P$ in each of the five cases. Tracing $\mc{D}$ upwards from the root, taking the left branch for \textsc{App} and \textsc{Conv}, and the right branch for \textsc{Weak}, $\Pi\mathsc{-rule}$, and $\lambda\mathsc{-rule}$, we alternate between instances of \textsc{Conv} and \textsc{Weak} before the $\sigma$-term $A$ is introduced in either an axiom, \textsc{Init}, $\Pi\mathsc{-rule}$, $\lambda\mathsc{-rule}$, or \textsc{App}; these five cases correspond respectively with the five cases for $\mc{D}$.

        Due to possible applications of \textsc{Weak} and \textsc{Conv}, the vertex of $\mc{D}$ right after $A$ is introduced has the form $\Gamma'\proves A\!:\!P'$ for some initial segment $\Gamma'$ of $\Gamma$ and some $\sigma$-term $P'$ such that $P'=_\beta P$. Then, inspecting the (unique) rule introducing $A$, we can conclude the premises of said rule and the desired form of $P'$.
    \end{proof}

    \begin{proof}[Proof of Theorem \ref{thm:subject_reduction}]
        We proceed by simultaneous induction on the derivation $\mc{D}:\Gamma\proves M\!:\!N$ when $M\rightarrow_\beta M'$ and $\Gamma\rightarrow_\beta\Gamma'$; the general case follows by iteration. We first prove (1), and split into cases with similar proofs.
        \begin{itemize}\small\vspace{-0.05in}
            \item If $\mc{D}$ ends with \textsc{Init}, then there is no redex in $M$. If $\mc{D}$ ends with \textsc{Conv}, then there are derivations $\mc{D}_1:\Gamma\proves M\!:\!N'$ and $\mc{D}_2:\Gamma\proves N'\!:\! s$ for some $s\in\mc{S}$ and some $\sigma$-term $N'$ such that $N'=_\beta N$. By $\mathrm{IH}_1$, we have $\Gamma\proves M'\!:\!N'$, on which \textsc{Conv} with $\mc{D}_2$ gives $\Gamma\proves M'\!:\!N$. The case when $\mc{D}$ ends with \textsc{Weak} is similar.
                \vspace{-0.05in}
            \item If $\mc{D}$ ends with $\Pi\mathsc{-rule}$, say with $M=\Pi x\!:\!B_1.B_2$, then the Stripping Lemma furnish some $(s_1,s_2,s_3)\in\mc{R}$ and derivations $\mc{D}_1:\Gamma\proves B_1\!:\!s_1$ and $\mc{D}_2:\Gamma,x\!:\!B_1\proves B_1\!:\!s_2$ such that $N=_\beta s_3$. By definition of $\rightarrow_\beta$, two cases occur: if there is a $\sigma$-term $B_1'$ such that $B_1\rightarrow_\beta B_1'$, then by $\mathrm{IH}_1$ on $\mc{D}_1$, we have $\mc{D}'_1:\Gamma\proves B_1'\!:\!s_1$. Moreover, $\mathrm{IH}_2$ on $\mc{D}_2$ gives $\mc{D}'_2:\Gamma,x\!:\!B_1'\proves B_2\!:\!s_2$, so applying $\Pi\mathsc{-rule}$ on $\mc{D}'_1$ and $\mc{D}'_2$ gives $\Gamma\proves(\Pi x\!:\!B_1'.B_2)\!:\!s_3$, on which \textsc{Conv} gives $\Gamma\proves(\Pi x\!:\!B_1'.B_2)\!:\!N$. The second case when $B_2\rightarrow_\beta B_2'$ for some $\sigma$-term $B_2'$ is the same (in fact, easier). The case when $\mc{D}$ ends with $\lambda\mathsc{-rule}$ is similar (and again has two subcases).
                \vspace{-0.05in}
            \item If $\mc{D}$ ends with \textsc{App}, say with $M=B_1\,B_2$, then reductions within either $B_1$ or $B_2$ are trivial. Thus, we can take $x\in\mc{V}\comp\dom\Gamma$ such that $B_1=\lambda x\!:\!A_1.A_2$, and assume $M=(\lambda x\!:\!A_1.A_2)B_2\rightarrow_\beta A_2[B_2/x]$. The Stripping Lemma then furnish $\sigma$-terms $C_1$ and $C_2$ such that $N=_\beta C_2[B_2/x]$ and derivations $\mc{D}_1:\Gamma\proves(\lambda x\!:\!A_1.A_2)\!:\!(\Pi x\!:\!C_1.C_2)$ and $\mc{D}_2:\Gamma\proves B_2\!:\!C_1$. Again, the Stripping Lemma applied to $\mc{D}_1$ then furnish $(s_1,s_2,s_3)\in\mc{R}$, a $\sigma$-term $C_2'$ such that $\Pi x\!:\!C_1.C_2=_\beta\Pi x\!:\!A_1.C_2'$, and derivations $\mc{E}_1:\Gamma\proves A_1\!:\!s_1$, $\mc{E}_2:\Gamma,x\!:\!A_1\proves C'_2\!:\!s_2$, and $\mc{E}_3:\Gamma,x\!:\!A_1\proves A_2\!:\!C_2'$. Observe that $A_1=_\beta C_1$, so \textsc{Conv} on $\mc{D}_2$ and $\mc{E}_1$ gives $\mc{D}_0:\Gamma\proves B_2\!:\!A_1$, and using the Substitution Lemma with $(\mc{D}_0,\mc{E}_2)$ and $(\mc{D}_0,\mc{E}_3)$ give $\mc{E}_2':\Gamma\proves C_2'[B_2/x]:s_2$ and $\mc{E}_3':\Gamma\proves A_2[B_2/x]\!:\!C_2'[B_2/x]$; note that $\Gamma[B_2/x]=\Gamma$ since $x\not\in\dom\Gamma$. Finally, since $C_2=_\beta C_2'$ and $N=_\beta C_2[B_2/x]$, applying \textsc{Conv} on $\mc{E}_2'$ and $\mc{E}_3'$ gives $\Gamma\proves A_2[B_2/x]\!:\!N$.
        \end{itemize}
        For (2), if the last rule of $\mc{D}$ is either \textsc{App}, \textsc{Conv}, $\Pi\mathsc{-rule}$, or $\lambda\mathsc{-rule}$, then we are done by $\mathrm{IH}_2$; indeed, $\Gamma$ is unchanged for \textsc{App} and \textsc{Conv}, and in $\Pi\mathsc{-rule}$ and $\lambda\mathsc{-rule}$, reductions take place within $\Gamma$. Suppose that the last rule of $\mc{D}$ is \textsc{Init} or \textsc{Weak}, so with the notation of Definition \ref{rules}, $\Gamma=\Gamma_0,x\!:\!A$ for some $\sigma$-term $A$ and $x\in\mc{V}$. If the reduction occurs within $\Gamma_0$, then we are done by $\mathrm{IH}_2$. Otherwise, $A\rightarrow_\beta A'$ for some $\sigma$-term $A$.
        \begin{itemize}\small\vspace{-0.05in}
            \item If $\mc{D}$ ends with \textsc{Init}, then $\Gamma_0\proves A\!:\!s$. Applying $\mathrm{IH}_1$, we have $\Gamma_0\proves A'\!:\!s$, and hence $\Gamma_0,x\!:\!A'\proves x\!:\!A'$ from \textsc{Init}. Since $A\rightarrow_\beta A'$, we see that $A=_\beta A'$, so $\Gamma_0,x\!:\!A'\proves x\!:\!A$ by \textsc{Conv}, as desired.
                \vspace{-0.05in}
            \item If $\mc{D}$ ends with \textsc{Weak}, then there are derivations $\mc{D}_1:\Gamma_0\proves A\!:\!s$ and $\mc{D}_2:\Gamma_0\proves B\!:\!C$. Applying $\mathrm{IH}_1$, we obtain a derivation $\mc{D}_1':\Gamma_0\proves A'\!:\!s$, and applying \textsc{Weak} on $\mc{D}_1'$ and $\mc{D}_2$ gives $\Gamma_0,x\!:\!A'\proves B\!:\!C$.
        \end{itemize}\vspace{-0.25in}
    \end{proof}

    \appendix

    \begin{bibdiv}
        \begin{biblist}*{labels={alphabetic}}
            \bibselect{bibliography}
        \end{biblist}
    \end{bibdiv}
\end{document}
