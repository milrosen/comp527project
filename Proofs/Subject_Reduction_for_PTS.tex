\documentclass[reqno]{amsart}
\input{preamble.sty}
\input{macros.sty}

\begin{document}
    \title{Subject Reduction for Pure Type Systems}
    \author{Zhaoshen Zhai}
    \address{Department of Mathematics and Statistics, McGill University, 805 Sherbrooke Street West, Montreal, QC, H3A 0B9, Canada}
    \email{zhaoshen.zhai@mail.mcgill.ca}
    \date{\today}
    \thanks{Notes for a project for \textsc{Comp527: Logic and Computation} taught by Professor Brigitte Pientka, with Charlotte Marchal, Dashiell Rich and Milton Rosenbaum.}

    \begin{abstract}
        Following \cite{SU06}, we give a detailed proof of \textit{subject reduction} for arbitrary \textit{pure type systems}, which abstract many of the basic constructs found in, say, the simply-typed $\lambda$-calculus ($\lambda_\rightarrow$), the polymorphic $\lambda$-calculus ($\lambda2$), the $\lambda$-calculus with type constructors ($\lambda\uline{\omega}$), and the $\lambda$-calculus with dependent types ($\lambda\mathbf{P}$).
    \end{abstract}

    \maketitle

    \subsection*{Introduction}

    Subject reduction is a crucial property of a type system that guarantees its `computational consistency' by ensuring that reductions of a well-typed expression remains well-typed, and which supports the slogan that `well-typed programs do not go wrong'. It is thus desirable that we can prove it uniformly across many different type systems, and this is the goal of the present note.

    To this end, we start from the beginning\footnote{As Professor Pientka would say: `We'll start slow'.} with the \textit{simply-typed $\lambda$-calculus} $\lambda_\rightarrow$, in which we prove subject reduction. We then progress to more complicated type systems (in particular, $\lambda2$, $\lambda\uline{\omega}$, and $\lambda\mathbf{P}$) to illustrate some concepts not present in $\lambda_\rightarrow$, and along the way, we also mention the \textit{$\lambda$-cube} to provide some motivation for \textit{pure type systems}, which abstract the constructs in all of the previous systems. Finally, we prove subject reduction for pure type systems. We will not discuss any of these systems in length, but refer the interested reader to \cite{SU06} for general type theory and \cite{Bar91} for actual applications of pure type systems.

    \section{The Simply-typed $\lambda$-calculus: $\lambda_\rightarrow$}

    Throughout, fix a countably infinite set $V$, whose element we call \textit{(propositional/$\lambda$-)variables}.

    \begin{definition}
        A \textit{simple type} is a propositional formula in the language $\l\{\rightarrow\r\}$; i.e., defined by the grammar $\tau\coloneqq x\,|\,\tau\rightarrow\tau$ where $x\in V$. We let $\Phi_\rightarrow$ denote the set of simple types.
    \end{definition}

    \begin{definition}
        A \textit{$\lambda$-term} is a string defined by the grammar\footnote{Sometimes, people write $\lambda x:\tau M$ instead of $\lambda x^\tau M$, but we find the former too hard to parse.} $M\coloneqq x\,|\,M\,M\,|\,(\lambda x^\tau M)$, where $\tau\in\Phi_\rightarrow$. We denote by $\Lambda$ the set of $\lambda$-terms. The set of \textit{free variables} of a $\lambda$-term $M$ is defined inductively by
        \begin{equation*}
            FV(x)\coloneqq\l\{x\r\},\ \ \ \ FV(\lambda x^\tau M)\coloneqq FV(M)\comp\l\{x\r\},\ \ \ \ FV(M\,N)\coloneqq FV(M)\cup FV(N).
        \end{equation*}
    \end{definition}

    \begin{remark}
        We always consider $\lambda$-terms under $\alpha$-conversion. Basically, we can freely change the bound variable $x$ in $\lambda x$ without modifying the term, but see \cite{SU06}*{Section 1.2} for the formal definition.
    \end{remark}

    \begin{definition}
        Let $M,N\in\Lambda$ and fix $x\in FV(M)$. The \textit{substitution} of $N$ for $x$ in $M$, written $M[N/x]$, is is the $\lambda$-term defined by induction on $M$; below, $y\neq x$ and $y\not\in FV(N)$.
        \begin{equation*}
            x[N/x]\coloneqq N,\ \ \ \ y[N/x]\coloneqq y,\ \ \ \ (P\,Q)[N/x]\coloneqq P[N/x]\,Q[N/x],\ \ \ \ (\lambda y^\tau P)[N/x]\coloneqq\lambda y^\tau.P[N/x].
        \end{equation*}
    \end{definition}

    \begin{definition}
        A \textit{context} is a finite set $\Gamma\coloneqq\l\{x_1:\tau_1,\dots,x_n:\tau_n\r\}$, where $x_i\in V$ are pairwise-distinct and each $\tau_i\in\Phi_\rightarrow$; that is, $\Gamma:V\parto\Phi_\rightarrow$ is a partial function, so we write $\Gamma(x)=\tau$ for $(x:\tau)\in\Gamma$ and we let
        \begin{equation*}
            \dom\Gamma\coloneqq\l\{x\in V\st\Gamma(x)=\tau\textrm{ for some }\tau\in\Phi_\rightarrow\r\}\ \ \ \ \textrm{and}\ \ \ \ \im\Gamma\coloneqq\l\{\tau\in\Phi_\rightarrow\st\Gamma(x)=\tau\textrm{ for some }x\in V\r\}.
        \end{equation*}
        A \textit{judgement} is a triple $\Gamma\proves M:\tau$ consisting of a context $\Gamma$, a $\lambda$-term $M$, and a simple type $\tau$.
    \end{definition}

    \begin{definition}
        We say that a judgement $\Gamma\proves M:\tau$ is \textit{derivable in $\lambda_\rightarrow$} if there is a finite tree of judgements rooted at $\Gamma\proves M:\tau$, whose leaves are instances of \textsc{Var}, and such that the children of each internal node is obtained from the rules \textsc{Abs} or \textsc{App} read bottom-up.
        \begin{equation*}
            \infer[\mathsc{Var}]{\Gamma,x:\tau\proves x:\tau}{}\ \ \ \ 
            \infer[\mathsc{Abs}]{\Gamma\proves(\lambda x^\sigma M):\sigma\rightarrow\tau}{\Gamma,x:\sigma\proves M:\tau}\ \ \ \ 
            \infer[\mathsc{App}]{\Gamma\proves(M\,N):\tau}{
                \Gamma\proves M:\sigma\rightarrow\tau &
                \Gamma\proves N:\sigma
            }
        \end{equation*}
        The rules \textsc{Var} and \textsc{Abs} can only be applied when $x\not\in\dom\Gamma$. We assert $\Gamma\proves M:\tau$ if it is derivable.
    \end{definition}

    \begin{lemma}[Generation Lemma 1]\label{lem:simply_typed_generation_1}
        Suppose that $\Gamma\proves M:\tau$.
        \begin{enumerate}
            \item If $M=x$, then $\Gamma(x)=\tau$.
            \item If $M=P\,Q$, then $\Gamma\proves P:\sigma\rightarrow\tau$ and $\Gamma\proves Q:\sigma$ for some $\sigma\in\Phi_\rightarrow$.
        \end{enumerate}
    \end{lemma}
    \begin{proof}
        Since the root of the derivation tree for $\Gamma\proves M:\tau$ determines the shape of $M$, we see that (1) follows from \textsc{Var} and (2) follows from \textsc{App}.
    \end{proof}

    \begin{lemma}[Variable Renaming]\label{lem:simply_typed_variable}
        If $\Gamma,x:\tau\proves M:\sigma$ and $y\not\in\dom\Gamma\cup\l\{x\r\}$, then $\Gamma,y:\tau\proves M[y/x]:\sigma$.
    \end{lemma}
    \begin{proof}
        By induction on the length of $M$. If $M=x$, then $\tau=\sigma$ and $M[y/x]=y$, and $\Gamma,y:\tau\proves y:\sigma$ by \textsc{Var}. If $M=z$ and $z\neq x$, then $M[y/x]=z$. Note that $\Gamma(z)=\sigma$ by Lemma \ref{lem:simply_typed_generation_1}.1, so $\Gamma,y:\tau\proves z:\sigma$ by \textsc{Var}.

        If $M=P\,Q$, then by Lemma \ref{lem:simply_typed_generation_1}.2, we have $\Gamma,x:\tau\proves P:\rho\rightarrow\sigma$ and $\Gamma,x:\tau\proves Q:\rho$ for some $\rho\in\Phi_\rightarrow$. By induction, we have $\Gamma,y:\tau\proves P[y/x]:\rho\rightarrow\sigma$ and $\Gamma,y:\tau\proves Q[y/x]:\rho$, on which \textsc{App} gives the desired.

        If $M=\lambda z^\rho N$, then $\Gamma,x:\tau\proves\lambda z^\rho N:\sigma$ is obtained by \textsc{Abs}, so it is of the form $\Gamma,x:\tau,w:\rho\proves P:\eta$ for some $\eta\in\Phi_\rightarrow$ and $w,P\in\Lambda$ such that $\lambda z^\rho N=\lambda w^\rho P$ and $\sigma=\rho\to\eta$. Up to $\alpha$-conversion, we can choose $w$ so that $w\neq y$. Then, since $|P|=|N|<|M|$, we have by induction that $\Gamma,y:\tau,w:\rho\proves P[y/x]:\eta$, on which \textsc{Abs} gives $\Gamma,y:\tau\proves\lambda w^\rho.P[y/x]:\sigma$. Now, $\lambda w^\rho.P[y/x]=(\lambda w^\rho P)[y/x]=(\lambda z^\rho N)[y/x]=M[y/x]$.
    \end{proof}

    \begin{lemma}[Generation Lemma 2]\label{lem:simply_typed_generation_2}
        If $\Gamma\proves\lambda x^\sigma M:\tau$ and $x\not\in\dom\Gamma$, then $\tau=\sigma\rightarrow\rho$ and $\Gamma,x:\sigma\proves N:\rho$ for some $\rho\in\Phi_\rightarrow$.
    \end{lemma}
    \begin{proof}
        As in the above proof, there exist $\rho\in\Phi_\rightarrow$ and $y,N\in\Lambda$ such that $\Gamma,y:\sigma\proves N:\rho$, $\lambda x^\sigma M=\lambda y^\sigma N$, and $\tau=\sigma\rightarrow\rho$. If $x=y$, we are done; otherwise, we have $N=M[y/x]$, so $\Gamma,y:\sigma\proves M[y/x]:\rho$, and finally substituting $x$ for $y$ gives $\Gamma,x:\sigma\proves M:\rho$ by Lemma \ref{lem:simply_typed_variable}, as desired.
    \end{proof}

    \begin{lemma}[Change of Context]\label{lem:simply_typed_change_of_context}
        If $\Gamma\proves M:\tau$ and $\Gamma(x)=\Gamma'(x)$ for all $x\in FV(M)$, then $\Gamma'\proves M:\tau$.
    \end{lemma}
    \begin{proof}
        By induction on $M$. If $M=x$, then $\Gamma'(x)=\Gamma(x)=\tau$ by Lemma \ref{lem:simply_typed_generation_1}.1, and hence $\Gamma'\proves x:\tau$ by \textsc{Var}. If $M=P\,Q$, then by Lemma \ref{lem:simply_typed_generation_1}.2, we have $\Gamma\proves P:\sigma\rightarrow\tau$ and $\Gamma\proves Q:\sigma$ for some $\sigma\in\Phi_\rightarrow$. By induction, we see that $\Gamma'\proves P:\sigma\to\tau$ and $\Gamma'\proves Q:\sigma$, on which \textsc{App} gives $\Gamma'\proves M:\tau$. Lastly, if $M=\lambda x^\sigma N$, we can choose $x\not\in\dom\Gamma\cup\dom\Gamma'$, so that $\tau=\sigma\rightarrow\rho$ and $\Gamma,x:\sigma\proves N:\rho$ by Lemma \ref{lem:simply_typed_generation_2}. By induction, we see that $\Gamma',x:\sigma\proves N:\rho$, on which \textsc{Abs} gives the desired as $\Gamma'\proves M:\tau$.
    \end{proof}

    We can think of the Change of Context lemma as a generalizing weakening as we can take $\Gamma'\coloneqq\Gamma,y:\sigma$ for $y\not\in FV(M)$, and this is exactly how we use it below.

    \begin{lemma}[Substitution Lemma]\label{lem:simply_typed_substitution}
        If $\Gamma,x:\sigma\proves M:\tau$ and $\Gamma\proves N:\sigma$, then $\Gamma\proves M[N/x]:\tau$.
    \end{lemma}
    \begin{proof}
        By induction on $M$. If $M=y$ and $x\neq y$, then $\Gamma(y)=\tau$ and $M[N/x]=y$, so that $\Gamma\proves y:\tau$ by \textsc{Var}. If $x=y$, then $\Gamma(x)=\sigma$ and $M[N/x]=N$, so $\tau=\sigma$ and $\Gamma\proves N:\sigma$ by assumption. If $M=P\,Q$, then by Lemma \ref{lem:simply_typed_generation_1}.2, we have $\Gamma,x:\sigma\proves P:\rho\rightarrow\tau$ and $\Gamma,x:\sigma\proves Q:\rho$ for some $\rho\in\Phi_\rightarrow$. By induction, we see that $\Gamma\proves P[N/x]:\rho\to\tau$ and $\Gamma\proves Q[N/x]:\rho$, on which \textsc{App} gives $\Gamma\proves M[N/x]:\tau$.

        Lastly, if $M=\lambda y^\eta M'$ where $y\not\in\dom\Gamma\cup\l\{x\r\}\cup FV(N)$, then by Lemma \ref{lem:simply_typed_generation_2}, there is some $\rho\in\Phi_\rightarrow$ such that $\tau=\eta\rightarrow\rho$ and $\Gamma,x:\sigma,y:\eta\proves M':\rho$. By Lemma \ref{lem:simply_typed_change_of_context}, we can weaken $\Gamma\proves N:\sigma$ to $\Gamma,y:\eta\proves N:\sigma$, so by induction\footnote{Note that our contexts are unordered, so we have exchange implicitly.} we have $\Gamma,y:\eta\proves M'[N/x]:\rho$, and we can apply \textsc{Abs} to get $\Gamma\proves M[N/x]:\tau$.
    \end{proof}

    \begin{definition}
        A relation $\esup$ on $\Lambda$ is \textit{compatible} if for any $M,N\in\Lambda$ with $M\esup N$, we have $MP\esup NP$ and $PM\esup PN$ for each $P\in\Lambda$, and $\lambda x^\tau M\esup\lambda x^\tau N$ for each $x\in V$ and $\tau\in\Phi_\rightarrow$.
    \end{definition}

    \begin{definition}
        The least compatible relation $\rightarrow_\beta$ on $\Lambda$ such that $(\lambda x^\tau M)N\rightarrow_\beta M[N/x]$ for all $M,N\in\Lambda$ is called \textit{$\beta$-reduction}. We say that $(\lambda x^\tau M)N$ is a \textit{$\beta$-redex} and that $M[N/x]$ arises by \textit{contracting} the redex.
    \end{definition}

    \begin{notation}
        For any relation $\rightarrow_\blob$ on a set $X$, we let $\twoheadrightarrow_\blob^+$ denote the transitive closure, let $\twoheadrightarrow_\blob$ denote the transitive and reflexive closure, and let $=_\blob$ denote the least equivalence relation containing $\twoheadrightarrow_\blob$.
    \end{notation}

    \begin{theorem}[Subject Reduction]
        If $\Gamma\proves M:\tau$ and $M\twoheadrightarrow_\beta N$, then $\Gamma\proves N:\tau$.
    \end{theorem}
    \begin{proof}
        If $M=(\lambda x^\sigma P)Q$ and $N=P[Q/x]$ for some $x\not\in\dom\Gamma$, then we have $\Gamma,x:\sigma\proves P:\tau$ and $\Gamma\proves Q:\sigma$ by Lemma \ref{lem:simply_typed_generation_1}.2 and \ref{lem:simply_typed_generation_2}, so $\Gamma\proves N:\tau$ by Lemma \ref{lem:simply_typed_substitution}. The general case follows by induction on $\twoheadrightarrow_\beta$.
    \end{proof}

    \section{The polymorphic $\lambda$-calculus: $\lambda2$}

    Throughout, fix two disjoint countable sets $V$ and $V_t$ of variables and \textit{type-variables}.

    \begin{definition}
        A \textit{polymorphic type} is a propositional formula in the language $\l\{\rightarrow,\fa\r\}$ over $V_\tau$, i.e., defined by the grammar $\tau\coloneqq x\,|\,\tau\rightarrow\tau\,|\,\fa p\,\tau$ where $x\in V$ and $p\in V_t$. Let $\Phi_2$ denote the set of polymorphic types.
    \end{definition}

    \begin{definition}
        A \textit{polymorphic term} is either a $\lambda$-term, a \textit{polymorphic abstraction} $\Lambda p\,M$ for $p\in V_t$, or a \textit{type application} $M\,\tau$ for $\tau\in\Phi_2$. That is, $M\coloneqq x\,|\,M\,M\,|\,\lambda x^\tau M\,|\,\Lambda p\,M\,|\,M\,\tau$ where $\tau\in\Phi_2$ and $p\in V_t$. We denote by $\Lambda_2$ the set of all polymorphic terms. The set of \textit{free variables} of a polymorphic term $M$ and a polymorphic type $\tau$ is defined inductively by extending $FV$ from before, and letting
        \begin{equation*}
            \begin{gathered}
                FV(\Lambda p\,M)\coloneqq FV(M)\comp\l\{p\r\},\ \ \ \ FV(M\,\tau)\coloneqq FV(M)\cup FV(\tau) \\
                FV(\tau\rightarrow\sigma)\coloneqq FV(\tau)\cup FV(\sigma),\ \ \ \ FV(\fa p\,\tau)\coloneqq FV(\tau)-\l\{p\r\}.
            \end{gathered}
        \end{equation*}
    \end{definition}

    \begin{definition}
        Let $\tau,\sigma\in\Phi_2$ and fix $p\in FV(\tau)$. The \textit{substitution} of $\sigma$ for $p$ in $\tau$, written $\tau[\sigma/p]$, is the polymorphic type defined by induction on $\tau$; below, $q\neq p$ and $q\not\in FV(\sigma)$.
        \begin{equation*}
            p[\sigma/p]\coloneqq\sigma,\ \ \ \ q[\sigma/p]\coloneqq q,\ \ \ \ (\tau_1\rightarrow\tau_2)[\sigma/p]\coloneqq\tau_1[\sigma/p]\rightarrow\tau_2[\sigma/p],\ \ \ \ (\fa q\,\tau)[\sigma/p]\coloneqq\fa q\,\tau[\sigma/p].
        \end{equation*}
    \end{definition}

    \begin{definition}
        For $M,N\in\Lambda_2$ and $x\in FV(M)$, we extend the substitution $M[N/x]$ by letting $(\Lambda p\,M)[N/x]\coloneqq\Lambda p\,M[N/x]$ for $p\not\in FV(N)$ and $(M\,\tau)[N/x]\coloneqq M[N/x]\,\tau$. For $p\in FV(M)$ and $\sigma\in\Phi_2$, we define the \textit{substitution} of $\sigma$ for $p$ in $M$, written $M[\sigma/p]$, as the polymorphic term below, where $q\not\in FV(\sigma)\cup\l\{p\r\}$.
        \begin{equation*}
            \begin{gathered}
                x[\sigma/p]\coloneqq x,\ \ \ \ (P\,Q)[\sigma/p]\coloneqq P[\sigma/p]\,Q[\sigma/p],\ \ \ \ (\lambda x^\tau M)[\sigma/p]\coloneqq\lambda x^{\tau[\sigma/p]}.M[\sigma/p],\\
                (M\,\tau)[\sigma/p]\coloneqq M[\sigma/p]\,\tau[\sigma/p],\ \ \ \ (\Lambda q\,M)[\sigma/p]\coloneqq\Lambda q\,M[\sigma/p].
            \end{gathered}
        \end{equation*}
    \end{definition}

    \begin{lemma}
        
    \end{lemma}

    \begin{theorem}[Subject Reduction for $\lambda2$]

    \end{theorem}

    \section{The $\lambda$-calculus with type constructors: $\lambda\uline{\omega}$}

    \begin{definition}
        
    \end{definition}

    \begin{lemma}
        
    \end{lemma}

    \begin{theorem}[Subject Reduction for $\lambda\uline{\omega}$]

    \end{theorem}

    \section{The $\lambda$-calculus with Dependent Types: $\lambda\mathbf{P}$}

    \begin{definition}
        
    \end{definition}

    \begin{lemma}
        
    \end{lemma}

    \begin{theorem}[Subject Reduction for $\lambda\mathbf{P}$]

    \end{theorem}

    \section{The $\lambda$-cube and beyond: Pure Type Systems}

    \begin{definition}
        
    \end{definition}

    \begin{lemma}
        
    \end{lemma}

    \begin{theorem}[Subject Reduction for Pure Type Systems]
        
    \end{theorem}

    \begin{bibdiv}
        \begin{biblist}*{labels={alphabetic}}
            \bibselect{bibliography}
        \end{biblist}
    \end{bibdiv}
\end{document}
