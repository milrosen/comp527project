\documentclass[reqno]{amsart}
\input{preamble.sty}
\input{macros.sty}

\begin{document}
    \title{Subject Reduction for Pure Type Systems}
    \author{Zhaoshen Zhai}
    \date{\today}
    \maketitle

    Throughout, fix a countably infinite set $V$, whose element we call \textit{variables}. For each of the following type systems, there will be a notion of `types' and `terms'. Once they are defined, we can speak of the following:

    \begin{definition*}
        A \textit{context} is a finite set $\Gamma\coloneqq\l\{x_1:\tau_1,\dots,x_n:\tau_n\r\}$ of pairs $(x_i:\tau_i)$, where each $x_i\in V$ and each $\tau_i$ is a `type'. If $(x:\tau)\in\Gamma$, we write $\Gamma(x)=\tau$, and we let
        \begin{equation*}
            \dom\Gamma\coloneqq\l\{x\in V\st(x:\tau)\textrm{ for some `type' }\tau\r\}\ \ \ \ \textrm{and}\ \ \ \ \im\Gamma\coloneqq\l\{\tau\textrm{ `type'}\st(x:\tau)\in\Gamma\textrm{ for some }x\in V\r\}.
        \end{equation*}
        A \textit{judgement} is a triple $\Gamma\proves M:\tau$ consisting of a context $\Gamma$, a `term' $M$, and a `type' $\tau$.
    \end{definition*}

    \section{The Simply-typed $\lambda$-calculus}

    \begin{definition}
        A \textit{simple type} is a propositional formula in the language $\rightarrow$.
    \end{definition}

    \begin{definition}
        A \textit{$\lambda$-term} is a string defined by the grammar $M\coloneqq x\,|\,M\,M\,|\,(\lambda x\,M)$. We denote by $\Lambda$ the set of $\lambda$-terms. The set of \textit{free variables} of a $\lambda$-term $M$ is defined inductively by
        \begin{equation*}
            FV(x)\coloneqq\l\{x\r\},\ \ \ \ FV(\lambda x\,M)\coloneqq FV(M)\comp\l\{x\r\},\ \ \ \ FV(MN)\coloneqq FV(M)\cup FV(N).
        \end{equation*}
    \end{definition}

    \begin{definition}
        We say that a judgement $\Gamma\proves M:\tau$ is \textit{derivable in $\lambda_\rightarrow$} if there is a finite tree of judgements rooted at $\Gamma\proves M:\tau$, whose leaves are instances of \textsc{Var}, and such that the children of each internal node is obtained from the rules \textsc{Abs} or \textsc{App} read bottom-up.
        \begin{equation*}
            \infer[\mathsc{Var}]{\Gamma,x:\tau\proves x:\tau}{}\ \ \ \ 
            \infer[\mathsc{Abs}]{\Gamma\proves(\lambda x\,M):\sigma\rightarrow\tau}{\Gamma,x:\sigma\proves M:\tau}\ \ \ \ 
            \infer[\mathsc{App}]{\Gamma\proves(M\,N):\tau}{
                \Gamma\proves M:\sigma\rightarrow\tau &
                \Gamma\proves N:\sigma
            }
        \end{equation*}
        The rules \textsc{Abs} and \textsc{App} can only be applied when $x\not\in\dom\Gamma$.
    \end{definition}

    \begin{lemma}[Generation Lemma for $\lambda_\rightarrow$]
        Suppose that\footnote{When we assert `$\Gamma\proves M:\tau$', we mean that it is derivable in the current type system under consideration.} $\Gamma\proves M:\tau$.
        \begin{enumerate}
            \item If $M=x$, then $\Gamma(x)=\tau$.
            \item If $M=PQ$, then $\Gamma\proves P:\sigma\rightarrow\tau$ and $\Gamma\proves Q:\sigma$ for some type $\sigma$.
            \item If $M=\lambda x\,N$ and $x\not\in\dom\Gamma$, then $\tau=\tau_1\rightarrow\tau_2$ and $\Gamma,x:\tau_1\proves N:\tau_2$.
        \end{enumerate}
    \end{lemma}
    \begin{proof}
        Since the root of the derivation tree for $\Gamma\proves M:\tau$ determines the shape of $M$, we see that (1) follows from \textsc{Var} and (2) follows from \textsc{App}. For (3), the child of the root must be obtained from \textsc{Abs} and is of the form $\Gamma,x':\tau_1\proves N':\tau_2$, where $\lambda x\,N=\lambda x'\,N'$. Clearly $\tau=\tau_1\rightarrow\tau_2$. Moreover, note that $N'=N[x'/x]$, so $\Gamma,x':\tau_1\proves N[x'/x]:\tau_2$, and finally substituting $x$ for $x'$ back gives $\Gamma,x:\tau_1\proves N:\tau_2$, as desired.
    \end{proof}

    \begin{lemma}[Change of Context]
        If $\Gamma\proves M:\sigma$ and $\Gamma(x)=\Gamma'(x)$ for all $x\in FV(M)$, then $\Gamma'\proves M:\sigma$.
    \end{lemma}
    \begin{proof}
        \TODO
    \end{proof}

    \begin{lemma}[Substitution Lemma for $\lambda_\rightarrow$]
        If $\Gamma,x:\tau\proves M:\sigma$ and $\Gamma\proves N:\tau$, then $\Gamma\proves M[N/x]:\sigma$.
    \end{lemma}
    \begin{proof}
        \TODO
    \end{proof}

    \begin{definition}
        A relation $\esup$ on $\Lambda$ is \textit{compatible} if for any $M,N\in\Lambda$ with $M\esup N$, we have $MP\esup NP$ and $PM\esup PN$ for each $P\in\Lambda$, and $\lambda x\,M\esup\lambda x\,N$ for each $x\in V$.

        The least compatible relation $\rightarrow_\beta$ on $\Lambda$ such that $(\lambda x\,M)N\rightarrow_\beta M[N/x]$ is called \textit{$\beta$-reduction}.
    \end{definition}

    \begin{notation}
        For any relation $\rightarrow_\blob$ on a set $X$, we let $\rightarrow_\blob^+$ denote the transitive closure, let $\rightarrow_\blob^\ast$ denote the transitive and reflexive closure, and let $=_\blob$ denote the least equivalence relation containing $\rightarrow_\blob$.
    \end{notation}

    \begin{theorem}[Subject Reduction for $\lambda_\rightarrow$]
        If $\Gamma\proves M:\sigma$ and $M\twoheadrightarrow_\beta N$, then $\Gamma\proves N:\sigma$.
    \end{theorem}
    \begin{proof}
        \TODO
    \end{proof}

    \section{The polymorphic $\lambda$-calculus: System $\mathbf{F}$}

    \begin{definition}
        
    \end{definition}

    \begin{lemma}
        
    \end{lemma}

    \begin{theorem}[Subject Reduction for $\mathbf{F}$]

    \end{theorem}

    \section{Dependent Types: $\lambda\mathbf{P}$}

    \begin{definition}
        
    \end{definition}

    \begin{lemma}
        
    \end{lemma}

    \begin{theorem}[Subject Reduction for $\lambda\mathbf{P}$]

    \end{theorem}

    \section{The $\lambda$-cube and beyond: Pure Type Systems}

    \begin{definition}
        
    \end{definition}

    \begin{lemma}
        
    \end{lemma}

    \begin{theorem}[Subject Reduction for Pure Type Systems]
        
    \end{theorem}
\end{document}
