\documentclass{beamer}
\usepackage{graphicx, tikz, tikz-cd}

\title{\normalsize{A general Subject Reduction for Pure Type Systems}\vspace{0.05in}}
\author{\small Charlotte Marchal, Dashiell Rich, Milton Rosenbaum, Zhaoshen Zhai}
\subtitle{\scriptsize{COMP527 $-$ Logic and Computation}}
\date{\footnotesize March 30, 2025}

\usetheme{Frankfurt}
\usecolortheme{seahorse}

% citations
% Logical Expressivism and Logical Relations, Lionel Shapiro
% "Thus, an account of what makes some vocabulary logical vocabulary must appeal to the fact that this vocabulary allows one to express what follows from what and what is incompatible with what. " https://philpapers.org/browse/logical-expressivism

\begin{document}\frame{\titlepage}
    \section{Lambda Cube}

    \begin{frame}{The Cube}
        \[\begin{tikzcd}[ampersand replacement=\&, row sep=3.4em]
            \& \lambda\omega \arrow[rr] \arrow[from=dd] \&\& \lambda C \\
            \lambda \textbf{2} \arrow[ur] \arrow[rr, crossing over] \&\& \lambda \textbf{P2} \arrow[ur] \& \\
            \& \lambda\underline{\omega} \arrow[rr] \&\& \lambda\textbf{P} \underline{\omega} \arrow[uu] \\
            \lambda_\rightarrow \arrow[uu] \arrow[rr] \arrow[ur] \&\& \lambda \textbf{P} \arrow[ur] \arrow[uu, crossing over]
        \end{tikzcd}\]
    \end{frame}

    \begin{frame}{Lots of Systems}
        In this diagram, eight systems are defined. The arrow represents $\subseteq$, so the weakest system is the simply typed lambda calculus, at the bottom left. Other named systems are $\lambda 2$, which is equivalent to system $\textbf{F}$, which is the polymorphic lambda calculus. $\lambda\omega$ is equivalent to $F\omega$, which was proposed by Girard, $\lambda P$ is equivalent to the "automath" language, and is sometimes called $LF$.
    \end{frame}

    \begin{frame}{Kinds of Dependencies}
        \[\begin{array}{c|l}
            (\star, \star) & \text{Terms can depend on terms} \\
            (\Box, \star) & \text{Terms can depend on types} \\
            (\star, \Box) & \text{Types can depend on terms} \\
            (\Box, \Box) & \text{Types can depend on types} \\
        \end{array}\]
    \end{frame}

    \begin{frame}{Parameterized \& Homogeneous}
        \[\begin{array}{l|cccc}
            \lambda\rightarrow & (\star, \star)\\
            \lambda \textbf{2} & (\star, \star) & (\Box, \star) \\
            \lambda \underline\omega & (\star, \star) && (\Box, \Box) \\
            \lambda \omega& (\star, \star) & (\Box, \star) & (\Box, \Box) \\
            \lambda\textbf{P} & (\star, \star) &&& (\star, \Box)\\
            \lambda \textbf{P2} & (\star, \star) & (\Box, \star)&& (\star, \Box) \\
            \lambda \textbf{P}\underline\omega & (\star, \star) && (\Box, \Box)& (\star, \Box) \\
            \lambda \textbf{P}\omega& (\star, \star) & (\Box, \star) & (\Box, \Box)&(\star, \Box) \\
        \end{array}\]
    \end{frame}

    \begin{frame}{Just change two two rules!}
        \[\begin{array}{lc}
            \Pi{-}rule & \dfrac{\Gamma \vdash A : s_1\quad \Gamma, x : A \vdash B : s_2}{\Gamma \vdash (\Pi x : A . B) : s_3} \\[1cm]
            \lambda{-}rule &\dfrac{\Gamma \vdash A : s_1\quad \Gamma, x : A \vdash b : B\quad \Gamma, x : A \vdash B : s_2}{\Gamma (\lambda x: A .b) : (\Pi x : A . B)}
        \end{array}\]
    \end{frame}

    \section{Pure type systems}

    \begin{frame}{What are they}
        In addition to $\Box$ and $\star$, we can add arbitrary sorts and axiom relations. A \textit{pure type system} $\lambda(S, A, R)$ can be described as 
        \begin{itemize}
            \item<1-> S : a set of sorts, for example $\star, \Box$
                \pause
            \item<2-> A : a set of axioms, $\star : \Box$
                \pause
            \item<3-> R : a set of rules, the valid ways to parameterize the $\lambda$ and $\Pi$ rules. $\lambda C$ has all of the pairs of sorts.
        \end{itemize}
        \pause
        We abbreviate `pure type system' by PTS.
    \end{frame}

    \begin{frame}{Even more systems}
        A surprising number of systems can be defined as PTSs:
        \begin{itemize}\small
            \item<1-> $\lambda C'$ = $(\star^t, \star^p, \Box), (\star^t : \Box, \star^p : \Box), (S^2, \text{ that is, all pairs})$
            \item<2-> $\lambda U$ = $[\star, \Box, \Delta], [\star : \Box, \Box : \Delta], [(\star, \star), (\Box, \star), (\Box, \Box), (\Delta, \Box), (\Delta, \star)]$
        \end{itemize}
    \end{frame}

    \section{Project}

    \begin{frame}{Mechanize Proof for Each PTS}
        Since so many systems are easily describable as PTSs, it would be nice to prove some properties about the systems in general. \\[0.05in]
        In our project, we plan to: 
        \begin{itemize}
            \item<1-> Prove Subject Reduction for arbitrary PTSs.
            \item<2-> Mechanize arbitrary PTSs in Beluga.
        \end{itemize}
    \end{frame}

    \begin{frame}{Tasks}
        \begin{itemize}
            \item<1-> Milton: Implementing PTSs in Beluga and mechanizing the proof of Subject Reduction.
            \item<2-> Everyone else: Prove Subject Reduction for PTSs on paper.
            \item<3-> Zhaoshen: Write up the extended abstract.
        \end{itemize}
    \end{frame}

    \begin{frame}{Obstacles}
        \begin{itemize}
            \item<1-> Programming in Beluga is \textit{hard}. \pause Like, really hard.
                \pause
            \item<2-> We need to first understand the proofs for System \textsf{F} and $\lambda\mathbf{P}$.
        \end{itemize}
    \end{frame}

    \begin{frame}{Timeline}
        For Milton:
        \begin{itemize}
            \item[Week 1] Mechanize basic properties of PTSs presented in the paper, mainly to make sure all the rules are correctly implemented.
                \pause
            \item[Week 2] Mechanize the main lemmas needed for Subject Reduction.
                \pause
            \item[Week 3] Mechanize the proof of Subject Reduction in Beluga.
        \end{itemize}

        \pause

        For everyone else:
        \begin{itemize}
            \item[Week 1] Learn some representative systems in the $\lambda$-cube and prove Subject Reduction for them.
                \pause
            \item[Week 2] Prove the main lemmas needed for Subject Reduction (mainly, the \textit{Generation Lemma}).
                \pause
            \item[Week 3] Prove Subject Reduction for arbitrary PTSs on paper.
        \end{itemize}
    \end{frame}
\end{document}
